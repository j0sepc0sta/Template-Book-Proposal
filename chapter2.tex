\chapter{Context and Technologies/Tools}\label{cap2}

\begin{adjustwidth}{2.5cm}{1cm}
\small This Chapter explains how to include Figures, Tables, and Equations in a paper.
\end{adjustwidth}

\section{Figures}
This section will cover how to place a Figure in a \LaTeX~document.
According to \cite{overleaf}, in Basic \LaTeX~you need the $graphicx$ package to include figures. That said, throughout the chapter, it's important to mention the meaning of the Figure, for example ``In Figure \ref{fig:leon} shows an example of a Figure will be illustrated''. Secondly, the caption of a Figure is \underline{always} after the figure. Sometimes it's necessary to place 2 Figures simultaneously, as illustrated in Figure \ref{fig:leon1}.
\begin{figure}[htpb]
    \centering
    \includegraphics[scale=0.4]{lion_large.png}
    \caption{Example of a figure}
    \label{fig:leon}
\end{figure}
\begin{figure}[htpb]
    \centering
    \subfigure[Figure 1]{\label{plot1}
    \includegraphics[scale=0.4]{lion_large.png}
    }\hspace{.5cm}
    \subfigure[Figure 2]{\label{plot2}
    \includegraphics[scale=0.4]{lion_large.png}
    }
    \caption{Figures Presented with $subfigure$ Package.}
    \label{fig:leon1}
\end{figure}\par
In Figure \ref{fig:leon1} each sub-figure has a sub-caption, in Figure \ref{fig:_side_to_side} two Figures will be illustrated with only one caption.
\begin{figure}[htpb]
    \centering
    \includegraphics[scale=0.4]{lion_large.png} \ \ \ \ \ \ \
    \includegraphics[scale=0.4]{lion_large.png}
    \caption{Example of Two Pictures, One Next to the Other.}
    \label{fig:_side_to_side}
\end{figure}

\section{Tables}
This section will cover how to place a Table in a \LaTeX~document.
According to \cite{overleaftables}, a Table is defined between the commands \verb|\begin{tabular}| and \verb|\end{tabular}|, an example of which is shown below.\par
\begin{table}[htpb]
    \renewcommand{\arraystretch}{1.5}
    \centering
    \caption{Centered Table}
    \label{tab1}
    \begin{tabular}{ccc}
        \hline
        Column & Column & Column \\
        \hline
        a & b & c \\
        d & e & f \\
        \hline
    \end{tabular}
\end{table}


After \verb|\begin{tabular}| is placed, between \verb|{}|, ccc, which indicates that the Table will have 3 columns, all centered. The number of letters indicates the number of columns and the letter their alignment: 
\begin{itemize}
    \item c for columns with text aligned centrally;
    \item l for columns with left-aligned text;
    \item r for columns with right-aligned text.
\end{itemize}\par
To indicate a column separation use \verb|&|. To indicate the number of lines use two
bars together, \verb|\|, which means a line break. The command \verb|\hline| is responsible for placing a horizontal line in the Table and the command \verb|\cline{-}| makes a horizontal line only between the indicated columns. To insert vertical lines, use \verb|| between the letters indicating the column alignment.
\begin{table}[htpb]
    \renewcommand{\arraystretch}{1.5}
    \centering
    \caption{Table with Left Alignment}.
    \label{tab2}
    \begin{tabular}{|l|cc|}
    \hline
    Column & Column & Column \\
    \hline \hline
    A & B & C \\
    \cline{2-3}
    D & E & F \\
    \hline
    \end{tabular}
\end{table}\par 

If a column contains a long text and there needs to be a line break within the cell, instead of using the letters c, l, or r, use \verb|p{}|, where \verb|{}| includes the size of the line.
\begin{table}[htpb]
    \renewcommand{\arraystretch}{1.5}
    \centering
    \caption{Table using $p\{\}$.}
    \label{tab3}
    \begin{tabular}{ccp{5cm}}
    \hline
    C & C & Text column \\
    \hline
    A & B & Large text will be typed here,
    but the cell width is fixed at 5 \si{\cm}.\\
\hline
\end{tabular}
\end{table}


It's possible to make the tables prettier by using \verb|\usepackage{booktabs}|, i.e. this package removes the \verb|\hline| and adds it: 
\begin{itemize}
    \item \verb|\toprule|, to the top line of the table;
    \item \verb|\midrule|, for the lines in the middle of the table;
    \item \verb|\bottomrule|, for the line below the table.
\end{itemize}
\begin{table}[htpb]
    \renewcommand{\arraystretch}{1.15}
    \centering
    \caption{Table Using $booktabs$ Package.}
    \label{table3}
    \begin{tabular}{llr}
        \toprule
        \multicolumn{2}{c}{Item} \\
        \cmidrule(r){1-2}
        Animal & Description & Price (\$)\\ \midrule
        Gnat & per gram & 13.65 \\
        & each & 0.01 \\
        Gnu & stuffed & 92.50 \\
        Emu & stuffed & 33.33 \\
        Armadillo & frozen & 8.99 \\
        \bottomrule
    \end{tabular}
\end{table}

\section{Equations}
In any mathematical formula, there are constants and variables. To modify the font and presentation of the elements according to their type, constant or variable, for example, $p''=max\{f(y),g(x)\}$ \footnote{Whenever you start an equation you must have the caption ``(2.1))'}.
\begin{equation}
    f_X(x) = \frac{1}{\sqrt{2 \pi \sigma^2}}and^{-\frac{(x-\mu)^2}{2\sigma^2}}
\end{equation}
\begin{equation*}
    f_X(x) = \frac{1}{\sqrt{2 \pi \sigma^2}}e^{-\frac{(x-\mu)^2}{2\sigma^2}}
\end{equation*}
\begin{center}
    $f_X(x) = \frac{1}{\sqrt{2 \pi \sigma^2}}e^{-\frac{(x-\mu)^2}{2\sigma^2}}$
\end{center}

\begin{subequations}
    \begin{align}
        f_X(x)&= \frac{1}{\sqrt{2 \pi \sigma^2}}and^{-\frac{(x-\mu)^2}{2\sigma^2}}\\
        f_X(x)&= \frac{1}{\sqrt{2 \pi \sigma^2}}and^{-\frac{(x-\mu)^2}{2\sigma^2}}\\
        f_X(x)&= \frac{1}{\sqrt{2 \pi \sigma^2}}and^{-\frac{(x-\mu)^2}{2\sigma^2}}\\
    \end{align}
\end{subequations}
For more information \cite{overleafsimbolos,simbolos}.